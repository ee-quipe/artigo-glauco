\documentclass[12pt]{article}

\usepackage{sbc-template}

\usepackage{graphicx,url}

\usepackage[brazil]{babel}   
\usepackage[utf8]{inputenc}  

     
\sloppy

\title{Pesquisa sobre o impacto de metodologias ágeis no processo de desenvolvimento de software em empresas da região de Joinville, SC.}

\author{Maruan Biasi\inst{1}, Ícaro Botelho\inst{1}, Gustavo Martins\inst{1}, Victor Matheus\inst{1}, Rafael Pereira\inst{1}}

\address{Centro Universitário Católica de Santa Catarina em Joinville\\ CEP 89203-005, Rua Visconde de Taunay, 427 - Centro, Joinville, SC - Brasil.}

\begin{document} 

\maketitle

\begin{abstract}

\end{abstract}
     
\begin{resumo} 

\end{resumo}
\section{Introdução}
\subsection{Objetivo Geral}

O artigo tem como objetivo investigar a utilização de metodologias ágeis no desenvolvimento de software, bem como aprofundar-se na influência desta prática dentro de empresa analisando os impactos positivos e negativos na rotina dos desenvolvedores.

\subsection{Objetivo Específico}

Com maior criticidade tem-se a intenção de apurar informações de modo qualitativo e quantitativo em ambientes empresariais na cidade de Joinville, Santa Catarina (SC) e cidades próximas.
\\Para a coleta de dados, em primeiro momento, será feita uma pesquisa de modo informal e anônimo usando ferramentas online. Essa fase nos ajudará a definir um caminho e foco para o estudo.
\\A adquirição dos dados qualitativos será feita através de entrevistas com outros acadêmicos do curso de Engenharia de Software e desenvolvedores de empresas da região.



\subsection{Justificativa}

A motivação desse artigo, é de elucidar a efetividade pratica das metodologias Ágeis dentro do ambiente real de trabalho em empresas de de Software na cidade de Joinville, SC e região.
\\Utilizando depoimentos e respostas coletadas pelos autores no publico alvo de estudantes do curso de Engenharia de Software.
\\O estudo analisa os impactos positivos e negativos relacionados a adoção de metodologias como \textit{Agile}, \textit{Lean}, e outras. O objetivo é possibilitar uma perspectiva mais clara e vasta sobre as vantagens e desvantagens destas prática, que prometem trazer grandes benefícios porém incluem custos e dificuldades que muitas vezes são menosprezadas.
\section{Metodologia de Pesquisa}
\subsection{Método de procedimento}
Para nossa base de pesquisa os autores decidiram por utilizar uma pesquisa voltada para o publico alvo, no caso desenvolvedores que venham a trabalhar em empresas de Joinville e região e que por ventura venham a utilizar a metodologia\textit{Agile}. 
\\Após esse levantamento de dados com a pesquisa, os autores pretendem realizar uma análise comparativa das respostas e gerar um mapeamento em um plano cartesiano alterado para está pesquisa e com base no \textbf{Fonte do Mapa e apresentar o mapa}.
\\Após isso concluirão com apontamentos e análises quanto os resultados encontrados dentro da pesquisa.
\subsection{Método de abordagem}
Para a etapa de pesquisa, os autores desenvolveram junto do professor orientador perguntas chave. O modelo adotado foi fornecido pelo professor orientador com base na \textbf{Fonte do Glauco}.
A pesquisa será feita de modo online, será coletado dados básicos dentro do questionário apenas para fins de validação da resposta. Mas todos os dados sensíveis serão segurados pelas leis da LGPD.
\\A análise será feita por aproximação de resultados, como as perguntas são objetivas as aproximações serão feitas de maneira que os resultados estejam todos aproximados.
\\Após validação e aproximação será feita a inclusão dos dados no plano cartesiano, seguindo as seguintes orientações e diretivas.
\\Após levantamento e aplicação em eixo cartesiano, os autores pretendem inferir conclusões sobre o trabalho, bem como elaborar hipóteses quanto ao benefício das praticas e aplicação da metodologia \textit{Agile}, nas empresas de Software de Joinville e região.
\subsection{Natureza do estudo}
O estudo em questão investiga a utilização de diferentes metodologias e a frequência de práticas de desenvolvimento de software em diferentes cidades de Santa Catarina. O estudo visualmente destaca a predominância de certas metodologias e a regularidade de interações entre equipes e com clientes. Por exemplo, a maior porcentagem de participantes é de Curitiba, Joinville e Jaraguá do Sul, sugerindo uma concentração de práticas ágeis nas cidades. A natureza deste estudo é, portanto, descritiva e analítica, fornecendo insights sobre tendências regionais no campo do desenvolvimento de software no Brasil.

\section{Pesquisa}

 A metodologia de desenvolvimento de software desempenha um papel crucial na garantia da qualidade e eficiência na criação de programas de computador. Nesta pesquisa, investigamos como as principais metodologias são aplicadas por equipes de desenvolvimento em diferentes cidades de Santa Catarina. Analisamos fatores como a frequência de troca de conhecimento, comunicação com clientes, realização de testes manuais, desenvolvimento de pipelines e interação entre os times.
\\ A abordagem ágil é amplamente adotada e valoriza a colaboração, adaptação e entrega contínua de valor ao cliente. Em cidades como Joinville e Curitiba, muitas equipes utilizam práticas ágeis, como Scrum e Kanban. Essas metodologias oferecem maior flexibilidade e permitem uma resposta rápida às necessidades dos clientes. No entanto, identificamos uma lacuna na comunicação entre os times internos.
\\O método Kanban é adotado em várias cidades e distritos, incluindo Joinville, Jaraguá do Sul e Pirabeiraba. Ele permite visualizar o fluxo de trabalho e priorizar tarefas de forma eficiente. No entanto, observamos que a comunicação com clientes e times internos não é tão eficaz quanto deveria ser. Isso resulta em problemas de qualidade nas aplicações e sistemas, com erros frequentes e retrabalhos.
\\Curitiba e Rio Negrinho mencionam o XGH, uma abordagem menos estruturada e mais informal. Embora seja menos comum, algumas equipes ainda a utilizam. No entanto, é importante destacar que a qualidade de entrega pode ser comprometida em troca de um desenvolvimento rápido e menos organizado.
A escolha da metodologia de desenvolvimento de software deve considerar a complexidade do projeto, o envolvimento do cliente e a experiência da equipe. As metodologias ágeis, como Scrum e Kanban, continuam sendo as mais populares, mas é fundamental abordar suas limitações e buscar aprimoramentos na comunicação e qualidade do trabalho.

\subsection{Trabalho}
Para o trabalho relacionado a matéria, foi desenvolvido um método de pesquisa que seja feito da melhor maneira possível e que se adéque a realidade dos acadêmicos, considerando suas atividades fora de sala de aula.
\\Em um determinado momento foi entendido pelos autores que a melhor maneira de adquirir respostas é a de encaminhar para outros acadêmicos um formulário virtual para a coleta de respostas. Para gerar e entender quais as melhores perguntas foi consultado o professor orientador. O mesmo possui materiais e referencias de como desenvolver tal pesquisa.
\\As perguntas por ele sugeridas serviram como base para as perguntas colocadas dentro da pesquisa.
\\Outro ponto que gerou debate dentro da pesquisa  entre os autores é o método para melhor filtrar e analisar os resultados colhidos. Bem como qual será a técnica utilizada para classificar as respostas dentro do plano cartesiano definido.
\\Utilizando o texto de \cite{Russo_Silva_Larieira_2021} utilizamos uma definição de gráfico em pizza para melhor visualização dos resultados encontrados pela pesquisa.
\subsection{Pesquisas complementares}
Perante a pesquisa aprofundada em artigos e materiais de estudo relevantes, foi elucidado o fato de que nem sempre metodologias ágeis são de todo boas, porém não devido a sua adoção em si, mas por conta de sua implementação, que por vezes forçada tende a ser problemática, fator esse que pode ser tão prejudicial quanto sua priorização equivocada. O problema não se encontra em princípios falhos dentro de uma prática como Scrum, ou Lean, mas sim na execução
ineficiente e apressada por parte de indivíduos despreparados para lidar com o recurso mais importante quando se lidando com empresas e projetos grandes.
\\Tudo se baseia no contexto no qual práticas ágeis começam a ser adotadas. O
cenário nem sempre apresenta as condições ideais para um início saudável no ciclo de vida de um projeto ágil. Líderes sobrecarregados, prazos apertados e orçamentos abaixo do necessário são alguns dos fatores que podem contribuir para que uma prática que visa contribuir para o sucesso de um projeto acabem sendo a causa de um desempenho abaixo do esperado ou até do previsto. Como dito anteriormente, pessoas são o foco de qualquer projeto, afinal, estes são o início e fim de qualquer plano de tarefa e funções desempenhadas. Desta forma, torna-se imperativo que qualquer processo vise de forma equilibrada resultados e entregas contínuas com feedbacks da própria equipe para melhorias dentro do processo em si.
\\Diante destas premissas, torna-se possível analisar o cerne do questionamento de se práticas ágeis são ou não uma garantia de sucesso, e se as mesmas podem ser causas da falha. O objetivo deste artigo é de esclarecer em detalhes quais fatores podem ser cruciais para a definição do futuro de um projeto, seja esta favorável ou não, e como as preocupações da gestão deste projeto devem levar em consideração uma visão abrangente de todos os recursos em jogo, não apenas em cena.
\section{Conclusão}
Os acadêmicos chegaram ao consenso de que a metodologia \textit{Agile} possui qualidades e eficiente quando bem aplicada. Contudo ao ser aplicada dentro das empresas, ela apresenta limitações e gera travas em processos e desenvolvimento.
\\Além disso entende-se que o escopo inicial da pesquisa acaba por limitar o resultado baseado na quantidade e diversidade das respostas. Ficando como sugestão para uma pesquisa futura aumentar as proporções da pesquisa, afim de observar se em outras regiões do pais e com maior variedade de respostas se obtém um resultado diferente.

\bibliographystyle{sbc}
\bibliography{sbc-template}
\nocite{*}

\end{document}
